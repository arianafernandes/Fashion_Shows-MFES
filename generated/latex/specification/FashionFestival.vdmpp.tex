\begin{vdmpp}[breaklines=true]
class FashionFestival
 
types
 public String = seq of char;
 public FestivalName = String;
 public FestivalDateBegin = String;
 public FestivalDateEnd = String;
 public FestivalLocal = String;

instance variables
 public  name : FestivalName := "";
 public  dateBegin : FestivalDateBegin := "";
 public  dateEnd : FestivalDateEnd := "";
 public  local : FestivalLocal := "";
 private  events: seq of Event := [];
 private  numberEvents: int := 0; 
 private  fashionUsers : set of FashionUser := {};
 private output : String := "";
 
(*@
\label{FashionFestival:20}
@*)
operations
 public FashionFestival : 
              FestivalName *
              FestivalDateBegin * 
              FestivalDateEnd *
              FestivalLocal            
              ==> FashionFestival
  FashionFestival(nm, di, df , lc) ==
  (
    name := nm;
    dateBegin := di;
    dateEnd := df;
    local := lc;    
    return self;
  );
  
(*@
\label{getName:36}
@*)
  --retorna os parametros da class event
  public pure getName : () ==> String
    getName() == return name;
(*@
\label{getDateBegin:39}
@*)
    
   public pure getDateBegin : () ==> String
     getDateBegin() == return dateBegin;
(*@
\label{getDateEnd:42}
@*)
  
  public pure getDateEnd : () ==> String
     getDateEnd() == return dateEnd;
(*@
\label{getLocal:45}
@*)
     
  public pure getLocal : () ==> String
     getLocal() == return local;
(*@
\label{getEvents:48}
@*)
  
  public pure getEvents : () ==> seq of Event
     getEvents() == return events;
     
(*@
\label{insertEvent:52}
@*)
 --Adiciona designer ao evento
 public insertEvent : Event ==> ()
  insertEvent(ev) ==
  (
    events := [ev] ^ events;
    numberEvents := numberEvents + 1;
  );
  
  
(*@
\label{insertFashionUser:61}
@*)
   --Adiciona designer ao evento
 public insertFashionUser: (FashionUser)  ==> ()
  insertFashionUser(us) ==
  (
    fashionUsers := fashionUsers union {us};
  );
  
(*@
\label{getNumberEvents:68}
@*)
 --retorna nr designers do evento
  public pure getNumberEvents : () ==> int
  getNumberEvents() == return numberEvents;
  
(*@
\label{getFashionUsers:72}
@*)
  --retorna utilizadores da aplica�ao
  public pure getFashionUsers : () ==> set of FashionUser
  getFashionUsers() == return fashionUsers;
(*@
\label{getNumberFashionUsers:75}
@*)
  
  public pure getNumberFashionUsers: () ==> nat
  getNumberFashionUsers() ==
  return (card fashionUsers);
(*@
\label{printFashionFestival:79}
@*)
  
  public printFashionFestival: () ==> String
  printFashionFestival() == (
  output := "Name: "^name^"\n"
       ^"Date Begin: "^dateBegin^"\n"
       ^"Date End: "^dateEnd^"\n"
       ^"Local: "^local^"\n";
  return output;
  );
  
end FashionFestival
\end{vdmpp}
\bigskip
\begin{longtable}{|l|r|r|r|}
\hline
Function or operation & Line & Coverage & Calls \\
\hline
\hline
\hyperref[FashionFestival:20]{FashionFestival} & 20&100.0\% & 2 \\
\hline
\hyperref[getDateBegin:39]{getDateBegin} & 39&100.0\% & 2 \\
\hline
\hyperref[getDateEnd:42]{getDateEnd} & 42&100.0\% & 2 \\
\hline
\hyperref[getEvents:48]{getEvents} & 48&100.0\% & 9 \\
\hline
\hyperref[getFashionUsers:72]{getFashionUsers} & 72&100.0\% & 2 \\
\hline
\hyperref[getLocal:45]{getLocal} & 45&100.0\% & 2 \\
\hline
\hyperref[getName:36]{getName} & 36&100.0\% & 4 \\
\hline
\hyperref[getNumberEvents:68]{getNumberEvents} & 68&100.0\% & 2 \\
\hline
\hyperref[getNumberFashionUsers:75]{getNumberFashionUsers} & 75&100.0\% & 1 \\
\hline
\hyperref[insertEvent:52]{insertEvent} & 52&100.0\% & 4 \\
\hline
\hyperref[insertFashionUser:61]{insertFashionUser} & 61&100.0\% & 2 \\
\hline
\hyperref[printFashionFestival:79]{printFashionFestival} & 79&100.0\% & 1 \\
\hline
\hline
FashionFestival.vdmpp & & 100.0\% & 33 \\
\hline
\end{longtable}

