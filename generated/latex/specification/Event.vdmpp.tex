\begin{vdmpp}[breaklines=true]
class Event
 
types
 public String = seq of char; 
 public EventName = String;
 public EventDate = String;
 public EventLocal = String;
 public EventTime = String;
 public EventDuration = String;
 public EventTheme = String;
 public EventGender = String;
  public EventCollection = String;

instance variables
 public  name : EventName := "";
 public  date : EventDate := "";
 public  local : EventLocal := "";
 public  time : EventTime := "";
 public  duration : EventDuration := "";
 public  theme : EventTheme := "";
 public  gender: EventGender := "";
 public  collection: EventCollection := "";
 public  output: String := "";
 
 --Lista de desfiles
 private runways: seq of Runway := [];
 
(*@
\label{Event:28}
@*)
operations
 public Event : 
         EventName * 
         EventDate *
         EventLocal *
         EventTime *
         EventDuration *
         EventTheme *
         EventGender *  
         EventCollection ==> Event
 Event(nm, dt, lc, hr , dr, tm, gr, cl) == (
  name := nm;
  date := dt;
  local := lc;
  time := hr;
  duration := dr;
  theme := tm;
  gender := gr;
  collection := cl;
  return self
 );
 
(*@
\label{getName:50}
@*)
 --retorna os parametros da class event
  public pure getName : () ==> String
    getName() == return name;
(*@
\label{getDate:53}
@*)
    
   public pure getDate : () ==> String
     getDate() == return date;
(*@
\label{getLocal:56}
@*)
     
  public pure getLocal : () ==> String
     getLocal() == return local;
(*@
\label{getTime:59}
@*)
     
  public pure getTime : () ==> String
     getTime() == return time;      
(*@
\label{getDuration:62}
@*)
     
  public pure getDuration : () ==> String
     getDuration() == return duration;
(*@
\label{getTheme:65}
@*)
     
  public pure getTheme : () ==> String
     getTheme() == return theme;
(*@
\label{getGender:68}
@*)
     
  public pure getGender : () ==> String
     getGender() == return gender;
(*@
\label{getCollection:71}
@*)
     
  public pure getCollection : () ==> String
     getCollection() == return collection;
(*@
\label{getRunways:74}
@*)
  
  public pure getRunways : () ==> seq of Runway
     getRunways() == return runways;
(*@
\label{getNumberRunways:77}
@*)
 
  public pure getNumberRunways : () ==> nat
     getNumberRunways() == return len runways;
(*@
\label{insertRunway:80}
@*)
     
  public insertRunway : Runway ==> ()
   insertRunway(r) ==
   (
     runways := runways ^  [r];
   );
(*@
\label{printFashionFestival:86}
@*)
(*@
\label{printEvent:86}
@*)
  
  public printEvent: () ==> String
  printEvent() == (
  output := "Event Name: "^name^"\n"
       ^"Date: "^date^"\n"
       ^"Time: "^time^"\n"
       ^"Theme: "^theme^"\n"
       ^"Gender: "^gender^"\n"
       ^"Collection: "^collection^"\n";
  return output;
  );
  
end Event
\end{vdmpp}
\bigskip
\begin{longtable}{|l|r|r|r|}
\hline
Function or operation & Line & Coverage & Calls \\
\hline
\hline
\hyperref[Event:28]{Event} & 28&100.0\% & 3 \\
\hline
\hyperref[getCollection:71]{getCollection} & 71&100.0\% & 2 \\
\hline
\hyperref[getDate:53]{getDate} & 53&100.0\% & 2 \\
\hline
\hyperref[getDuration:62]{getDuration} & 62&100.0\% & 1 \\
\hline
\hyperref[getGender:68]{getGender} & 68&100.0\% & 2 \\
\hline
\hyperref[getLocal:56]{getLocal} & 56&100.0\% & 1 \\
\hline
\hyperref[getName:50]{getName} & 50&100.0\% & 14 \\
\hline
\hyperref[getNumberRunways:77]{getNumberRunways} & 77&100.0\% & 3 \\
\hline
\hyperref[getRunways:74]{getRunways} & 74&100.0\% & 5 \\
\hline
\hyperref[getTheme:65]{getTheme} & 65&100.0\% & 2 \\
\hline
\hyperref[getTime:59]{getTime} & 59&100.0\% & 2 \\
\hline
\hyperref[insertRunway:80]{insertRunway} & 80&100.0\% & 4 \\
\hline
\hyperref[printEvent:86]{printEvent} & 86&100.0\% & 1 \\
\hline
\hyperref[printFashionFestival:86]{printFashionFestival} & 86&100.0\% & 1 \\
\hline
\hline
Event.vdmpp & & 100.0\% & 43 \\
\hline
\end{longtable}

