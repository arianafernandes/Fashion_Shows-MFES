\begin{vdmpp}[breaklines=true]
class Event
 
types
 public String = seq of char; 
 public EventName = String;
 public EventDate = String;
 public EventLocal = String;
 public EventTime = nat;
 public EventDuration = nat;
 public EventTheme = String;
 public EventGender = <Homem> | <Mulher> | <Unisexo>;
  public EventCollection = <Outono_Inverno> | <Primavera_Verao>;

instance variables
 public  name : EventName := "";
 public  date : EventDate := "";
 public  local : EventLocal := "";
 public  time : EventTime := 0;
 public  duration : EventDuration := 0;
 public  theme : EventTheme := "";
 public  gender: EventGender := <Homem>;
 public  collection: EventCollection := <Outono_Inverno>;
 --Lista de designers do evento
 private  designers: seq of Designer := [];
 private  numberDesigners: int := 0;
 --Lista de modelos do evento
 private  models: seq of Model := [];
 private  numberModels: int := 0;
 
operations
(*@
\label{Event:31}
@*)
 public Event : 
         EventName * 
         EventDate *
         EventLocal *
         EventTime *
         EventDuration *
         EventTheme *
         EventGender *  
         EventCollection ==> Event
 Event(nm, dt, lc, hr , dr, tm, gr, cl) == (
  name := nm;
  date := dt;
  local := lc;
  time := hr;
  duration := dr;
  theme := tm;
  gender := gr;
  collection := cl;
  return self
 );
 
 --retorna os parametros da class event
(*@
\label{getName:53}
@*)
  public pure getName : () ==> String
    getName() == return name;
    
(*@
\label{getDate:56}
@*)
   public pure getDate : () ==> String
     getDate() == return date;
     
(*@
\label{getLocal:59}
@*)
  public pure getLocal : () ==> String
     getLocal() == return local;
     
(*@
\label{getTime:62}
@*)
  public pure getTime : () ==> nat
     getTime() == return time;      
     
(*@
\label{getDuration:65}
@*)
  public pure getDuration : () ==> nat
     getDuration() == return duration;
     
(*@
\label{getTheme:68}
@*)
  public pure getTheme : () ==> String
     getTheme() == return theme;
     
(*@
\label{getGender:71}
@*)
  public pure getGender : () ==> EventGender
     getGender() == return gender;
     
(*@
\label{getCollection:74}
@*)
  public pure getCollection : () ==> EventCollection
     getCollection() == return collection;
  
(*@
\label{getModels:77}
@*)
  public pure getModels : () ==> seq of Model
     getModels() == return models;
     
(*@
\label{getDesigners:80}
@*)
  public pure getDesigners : () ==> seq of Designer
     getDesigners() == return designers;
 
 --DESIGNERS DO EVENTO 
 --Adiciona designer ao evento
(*@
\label{insertDesigner:85}
@*)
 public insertDesigner : Designer ==> ()
  insertDesigner(dg) ==
  (
    numberDesigners := numberDesigners + 1;
    designers := designers ^  [dg];
  );
  
 --retorna nr designers do evento
(*@
\label{getNumberDesigners:93}
@*)
  public pure getNumberDesigners : () ==> int
  getNumberDesigners() == return numberDesigners;
  
  --MODELOS DO EVENTO
  --Adiciona model ao evento
(*@
\label{insertModel:98}
@*)
 public insertModel : Model ==> ()
  insertModel(md) ==
  (
    numberModels := numberModels + 1;
    models := models ^  [md];
  );
  
  --retorna nr designers do evento
(*@
\label{getNumberModels:106}
@*)
  public pure getNumberModels : () ==> int
  getNumberModels() == return numberModels;
 
end Event
\end{vdmpp}
\bigskip
\begin{longtable}{|l|r|r|r|}
\hline
Function or operation & Line & Coverage & Calls \\
\hline
\hline
\hyperref[Event:31]{Event} & 31&100.0\% & 3 \\
\hline
\hyperref[getCollection:74]{getCollection} & 74&100.0\% & 1 \\
\hline
\hyperref[getDate:56]{getDate} & 56&100.0\% & 1 \\
\hline
\hyperref[getDesigners:80]{getDesigners} & 80&100.0\% & 2 \\
\hline
\hyperref[getDuration:65]{getDuration} & 65&100.0\% & 1 \\
\hline
\hyperref[getGender:71]{getGender} & 71&100.0\% & 1 \\
\hline
\hyperref[getLocal:59]{getLocal} & 59&100.0\% & 1 \\
\hline
\hyperref[getModels:77]{getModels} & 77&100.0\% & 5 \\
\hline
\hyperref[getName:53]{getName} & 53&100.0\% & 9 \\
\hline
\hyperref[getNumberDesigners:93]{getNumberDesigners} & 93&100.0\% & 2 \\
\hline
\hyperref[getNumberModels:106]{getNumberModels} & 106&100.0\% & 3 \\
\hline
\hyperref[getTheme:68]{getTheme} & 68&100.0\% & 1 \\
\hline
\hyperref[getTime:62]{getTime} & 62&100.0\% & 1 \\
\hline
\hyperref[insertDesigner:85]{insertDesigner} & 85&100.0\% & 2 \\
\hline
\hyperref[insertModel:98]{insertModel} & 98&100.0\% & 5 \\
\hline
\hline
Event.vdmpp & & 100.0\% & 38 \\
\hline
\end{longtable}

